\documentclass[a4paper]{article}
\usepackage{fullpage}
\usepackage{amssymb}
\usepackage{amsmath}
\usepackage{graphicx}


\title{Coursework: Divide and Conquer -- Report}
\author{Alexandre Dalyac \and Hamza Haider \and Kiryl Trembovolski}

\begin{document}
\maketitle

\section*{Latex Examples}

\begin{figure}[h]

\includegraphics[width=0.48\linewidth]{example_plot1.png}
\includegraphics[width=0.48\linewidth]{example_plot2.png}

\caption{Example for embedding plots in Latex.}
\label{fig:examples}
\end{figure}


\section{Recurrences}

\subsection{Substitution Method}

Recursion trees allow you to analyse recurrences to obtain a guess for the substitution method. A closely related method is to expand out the recurrence a few times, until a pattern emerges. We call this the expansion method. An example of how to use the expansion method is given below.\\
\newline
Use the expansion method to guess an upper bound and the substitution method to verify your guess:

\begin{itemize}

\item Example: $T(n) = 2T(n/2) + n$

Introduce constants $c$ and expand the recurrence: \\
\begin{align*}
T(n) & \leq 2T(n/2) + cn \\
~ & \leq 2[2T(n/4) + cn/2] + cn = 4T(n/4)+2cn \\
~ & \leq 4[2T(n/8) + cn/4] + 2cn = 8T(n/8)+3cn \\
~ & \leq 8[2T(n/16) + cn/8] + 3cn = 16T(n/16)+4cn \\
~ & \vdots
\end{align*}

A pattern is emerging. The general term is $T(n) \leq 2^k T(n/2^k) + kcn$. Plugging in $k=\lg n$ (the height of the recursion tree), we get $T(n) \leq nT(1) + cn \lg n = O(n \lg n)$.\\
\newline
Now we verify $O(n \lg n)$ using the substitution method:\\
\newline
Prove that
\begin{align*}
T(n) \leq c(n \lg n)
\end{align*}
Assume that
\begin{align*}
T(\frac{n}{2}) \leq c(\frac{n}{2} \lg \frac{n}{2})
\end{align*}
By substitution
\begin{align*}
T(n) & \leq 2(c(\frac{n}{2} \lg \frac{n}{2})) +  n \\
~ & = cn \lg \frac{n}{2} + n \\
~ & = cn \lg n - cn \lg 2 + n \\
~ & \leq cn \lg n \quad \text{holds for} \, c \geq 1 \\
\end{align*}

\item $T(n) = 3T(n/2) + n$ 

This is a linear, finite history, constant coefficient recurrence. In order to apply the substitution method (also known as \textit{iteration method}), let us write down a few terms from this recursion that we will use below in backsubstitution: \\ 

$T(n/2) = 3T(n/4) + n/2 = 3T(n/2^2) + n/2^1$ \\
$T(n/4) = 3T(n/8) + n/4 = 3T(n/2^3) + n/2^2$ \\

In this simple case two terms turn out to be sufficient to spot an emerging pattern: \\
$T(n) = 3T(n/2) + n = 3[3T(n/2^2) + n/2^1] + n = $\\
$3^2T(n/2^2) + 3n/2^1 + n = 3^2[3T(n/2^3)+n/2^2] + 3n/2 + n = $\\
$3^3T(n/2^3) + 3^2n/2^2 + 3n/2^1 + n = $ \\
  \vdots \\
$3^{log_2n}T(1) + n\sum \limits_{i=0}^{log_2{n}-1} (3/2)^i =  n^{log_2{3}} + n\sum \limits_{i=0}^{log_2{n}-1} (3/2)^i $ \\
\\
Now, it is trivial to calculate the finite sum in the expression above: \\
$n\sum \limits_{i=0}^{log_2{n}-1} (3/2)^i = n(\frac{1 - (3/2)^{log_2n}}{1 - (3/2)}) = $\\
$2n((3/2)^{log_2n}-1) = 2n(\frac{3^{log_2n}}{n}) - 2n = 2n^{log_23} - 2n$ \\
Substitution yields: \\ \\
$T(n) = n^{log_2{3}} + 2n^{log_23} - 2n$ \\

Then clearly there exists a positive constant $C$ such that for all sufficiently large $n$ we will have: \\
\\
$T(n) = n^{log_2{3}} + 2n^{log_23} - 2n \leq Cn^{log_2{3}}  $
\\
Recalling the definition of the big O, we can immediately write: \\
$T(n) = O(n^{log_23})$

Now, let us verify the found complexity using the substitution method. We want to prove that: \\ \\
$T(n) \leq Mn^{log_23}$, where $M$ is a constant\\ \\
We assume that the following holds: \\ \\
$T(n/2) \leq D(n/2)^{log_23}$, where $D$ is a constant \\ \\
Substitution yields: \\ \\
$T(n) \leq 3D(n/2)^{log_23} + n = \frac{3Dn^{log_23}}{2^{log_23}} + n = Dn^{log_23} + n$ \\ \\
Now, it is intuitively clear that because $log_23 > 1$ and since polynomial function grows significantly faster than a linear function, there will exist a constant $M$ such that for all sufficiently large $n$ the following will hold: \\ \\
$T(n) \leq Dn^{log_23} + n \leq Mn^{log_23}$ \\ \\
However, if we want to be rigorous it is quite trivial to prove it. We want to find $M$ such that: \\ \\
$Dn^{log_23} + n \leq Mn^{log_23}$ \\ \\
Solving for $M$: \\ \\
$D + \frac{n^{log_22}}{n^{log_23}} \leq M \Leftrightarrow D  + n^{log_22 - log_23}  \leq M \Leftrightarrow D + n^{log_22/3}  \leq M$ \\ \\
Now, clearly since $2/3 < 1$, we can state that $\forall n: n>1$ $\Rightarrow$ $n^{log_22/3} < 1$.  \\
$\therefore$ $\forall n>1$, let $M = D + 1$ 

\item $T(n) = T(n/2) + n^2$

This is a linear, finite history, constant coefficient recurrence. Similarly, let us write down a few terms from this recursion that we will use below in backsubstitution: \\

$T(n/2) = T(n/4) + n^2/2^2$ \\
$T(n/4) = T(n/8) + n^2/2^4$\\
$T(n/8) = T(n/16) + n^2/2^6$ \\ \\
Substituting back: \\
$T(n) = T(n/2^1) + n^2 = T(n/2^2) + n^2/2^2 + n^2 = T(n/2^3) + n^2/2^4 + n^2/2^2 + n^2 = $\\
$T(n/2^4) + n^2/2^6 + n^2/2^4 + n^2/2^2 + n^2 = $ \\
  \vdots \\
$T(1) + n^2\sum\limits_{i=0}^{log_2n}(1/2)^i \leq n^2\sum\limits_{i=0}^{\infty}(1/2)^i = 2n^2$ \\ \\
Again, we can always find a constant $M$ such that for all sufficiently large $n$ we will have that $2n^2 \leq Mn^2$; and this is in turn just the definition of the Big O and hence: \\ \\
$T(n) = O(n^2)$ \\ \\
Let us verify the found complexity using the substitution method. Out aim is to prove that: \\ \\
$T(n) \leq Mn^2$, where $M$ is a constant\\ \\
We assume that the following holds: \\ \\
$T(n/2) \leq Dn^2/4$, where $D$ is a constant \\ \\
Substitution yields: \\ \\
$T(n) \leq \frac{Dn^2}{4} + n^2 = n^2(\frac{D}{4} + 1)$ \\ \\
It is not difficult to see that $\exists M:$ for all sufficiently large $n$, $n^2(\frac{D}{4} + 1) \leq Mn^2$ \\ \\
In fact, dividing both sides of the inequality by $n^2$, we can see that the choice of $M$ does not depend on $n$ at all! For any given $D$ we choose $M$ such that:\\ \\
$\frac{D}{4} + 1 \leq M$


\item $T(n) = 4T(n/2 + 2) + n$

This is a linear, finite history, constant coefficient recurrence. Let us expand it: \\

$T(n) = 4T(n/2 +2) + n = 4(4T(n/2^2 + 1 + 2) + (n/2 + 2)) + n = $ \\
$4^2T(n/2^2 +1 + 2) + 4(n/2 + 2) + n = $ \\
$4^2(4T(n/2^3 + 1/2 + 1 + 2) + (n/2^2 + 1 + 2)) + 4(n/2 + 2) + n = $ \\
$4^3T(n/2^3 + 1/2 + 1 + 2) + 4^2(n/2^2 + 1 + 2) + 4(n/2 + 2) + n = $ \\
$4^3(4T(n/2^4 + 1/4 + 1/2 + 1 + 2) + (n/2^3 + 1/2 + 1 + 2)) + 4^2(n/2^2 + 1 + 2) + 4(n/2 + 2) + n = $ \\
$4^4T(n/2^4 + 1/4 + 1/2 + 1 + 2) + 4^3(n/2^3 + 1/2 + 1 + 2) + 4^2(n/2^2 + 1 + 2) + 4(n/2 + 2) + n = $ \\
\vdots \\
$4^{log_2n}T(1 + \sum\limits_{i =-1}^{log_{2}n - 2} (1/2)^i) + n\sum\limits_{j=0}^{log_2n - 1} (2^j) + \sum\limits_{p=-1}^{log_2n - 3} \sum \limits_{k = -1}^{p} (1/2)^k = $ \\
$ n^2T(1+2\sum\limits_{i=0}^{log_2n - 1} (1/2)^i) + n(\frac{1-2^{log_2n}}{1-2}) + 2\sum\limits_{p=-1}^{log_2n - 3} \sum \limits_{k = 0}^{p+1} (1/2)^k \leq $ \\
$n^2T(1 + 2\sum\limits_{i=0}^{\infty} (1/2)^i) + n - n^2 + 2\sum\limits_{p=0}^{log_2n - 2} \sum \limits_{k = 0}^{p} (1/2)^k = $ \\
$n^2(T(const) - 1) + n + 2\sum\limits_{p=0}^{log_2n - 2} \frac{(1 - (1/2)^{p+1})}{1 - 1/2} = $ \\
$n^2(T(const) - 1) + n + 4\sum\limits_{p = 0}^{log_2n - 2} - \sum \limits_{p=0}^{log_2n - 2} (1/2)^{p+1}$ \\ \\
Since in the last term in the expression above the sequence under the summation sign takes only positive values, we can write: \\ \\
$n^2(T(const) - 1) + n + 4\sum\limits_{p = 0}^{log_2n - 2} - \sum \limits_{p=0}^{log_2n - 2} (1/2)^{p+1} \leq$ \\
$n^2(T(const) - 1) + n + 4log_2n$ \\
$n^2$ term in the expression above grows much faster than both linear and log terms; hence, it is always possible to find a constant $M$ such that for all sufficiently large $n$: \\ \\
$n^2(T(const) - 1) + n + 4log_2n \leq Mn^2$ \\
$\therefore T(n) = O(n^2)$ \\ \\

Again, let us verify the complexity we have found using the substitution method. What we want to prove is that: \\ \\
$T(n) \leq Mn^2$, where $M$ is a positive constant\\ \\
We assume that the following holds: \\ \\
$T(n/2 + 2) \leq D(n/2 + 2)^2$, where $D$ is a positive constant \\ \\
Substitution yields: \\ \\
$T(n) \leq 4D(n/2 + 2)^2 + n = 4D(n^2/4 + 2n + 4) + n = Dn^2 + 8Dn + 16D + n = $\\
$Dn^2 + n(8D+1) + 16D$ \\ \\
Similarly to the previous problems, what we want to show is that for all sufficiently large $n$ we can find some constant $M$ such that: \\ \\ 
$Dn^2 + n(8D+1) + 16D \leq Mn^2$ \\ \\
Dividing both sides by $n^2$: \\ \\
$D + \frac{8D+1}{n} + \frac{16D}{n^2} \leq M$ \\ \\
Since in order to show that $T(n) = O(n^2)$ it suffices to show that a constant $M$ exists (i.e. we don't need to find the smallest one), it is perfectly legitimate to state the following: \\ \\
$\forall$ $n: n>8D+1$, where  $D$ is a positive constant, and for $M: M=D+2$, the following holds: \\ \\
$D + \frac{8D+1}{n} + \frac{16D}{n^2} \leq M$ \\ \\
Since the second and third terms for the given values of $n$ will be less than 1.

\item $T(n) = 2T(n-1) + 1$

This is a linear, finite history, constant coefficient recurrence. Let us expand it:  \\ \\
$T(n) = 2T(n-1) + 1 = 2(2T(n-2)+1)+1 = 2^2T(n-2)+2+1 = $ \\
$2^2(2T(n-3)+1)+2+1 = 2^3T(n-3) + 2^2 + 2 + 1 = $ \\
$2^4T(n-4) + 2^3 + 2^2 + 2 + 1 = $ \\
\vdots \\
$2^nT(1) + \sum\limits_{i=0}^{n-1}2^i = 2^nT(1) + \frac{1-2^n}{1-2} = 2^nT(1) + 2^n - 1= $ \\
$2^n(T(1) + 1) - 1$ \\ \\
Now, trivially there exists a constant $M$ such that for all sufficiently large $n$: \\ \\
$2^n(T(1) + 1) - 1 \leq M2^n$ \\
$\therefore T(n) = O(2^n)$
 
In this case the verification of the found complexity turns out to be truly trivial. Our aim is to prove that: \\ \\
$T(n) \leq M2^n$, where $M$ is a positive constant \\ \\
We assume that the following holds: \\ \\
$T(n-1) \leq D\frac{2^n}{2}$ \\ \\
Substitution yields: \\ \\
$T(n) \leq 2D\frac{2^n}{2} + 1 = D2^n + 1$ \\ \\
Taking $M = D + 2$ we get that $\forall n: n>0$: \\ \\
$D2^n + 1 \leq M2^n$
\item $T(n) = T(n-1) + T(n/2) + n$
	

\end{itemize}

\subsection{Master Method}

Use the master method to give tight asymptotic $\Theta$ bounds:

\begin{itemize}

\item Example: $T(n) = 2T(n/2) + n$

Cost at leaves: $\Theta(n^{\log_b a}): n^{\log_2 2}=n=\Theta(n)$ \\
Cost per depth: $f(n)=n$ \\
Case 2: $f(n) = \Theta(n^{\log_b a})$: $f(n) = \Theta(n)$ \\
Solution: $\Theta(n^{\log_b a} \lg n) = \Theta(n \lg n)$

\item $T(n) = 2T(n/4) + 1$

Insert your solution here

\item $T(n) = 2T(n/4) + n$

Insert your solution here

\item $T(n) = 2T(n/2 + 17) + n$

Insert your solution here

\item $T(n) = 4T(n/2) + 2^n$

Insert your solution here

\item $T(n) = T(3n/4) + \sqrt{n}$

Insert your solution here

\end{itemize}

\section{Sorting}

\subsection{Plots}

Insert your plots here

\subsection{Discussion}

Insert your discussion here

\section{Integer Multiplication}

\subsection{Plots}

\noindent Insert your plots here

\subsection{Discussion}

\noindent Insert your discussion here

\subsection{Recurrences}

\subsubsection{Recursive Multiplication}

\noindent Insert your solution here

\subsubsection{Karatsuba Multiplication}

\noindent Insert your solution here

\end{document}
