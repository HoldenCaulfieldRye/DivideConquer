\documentclass[a4paper]{article}
\usepackage{fullpage}
\usepackage{amssymb}
\usepackage{amsmath}
\usepackage{graphicx}


\title{Coursework: Divide and Conquer -- Report}


\begin{document}
\maketitle

\section*{Latex Examples}

\begin{figure}[h]

\includegraphics[width=0.48\linewidth]{example_plot1.png}
\includegraphics[width=0.48\linewidth]{example_plot2.png}

\caption{Example for embedding plots in Latex.}
\label{fig:examples}
\end{figure}


\section{Recurrences}

\subsection{Substitution Method}

Recursion trees allow you to analyse recurrences to obtain a guess for the substitution method. A closely related method is to expand out the recurrence a few times, until a pattern emerges. We call this the expansion method. An example of how to use the expansion method is given below.\\
\newline
Use the expansion method to guess an upper bound and the substitution method to verify your guess:

\begin{itemize}

\item Example: $T(n) = 2T(n/2) + n$

Introduce constants $c$ and expand the recurrence: \\
\begin{align*}
T(n) & \leq 2T(n/2) + cn \\
~ & \leq 2[2T(n/4) + cn/2] + cn = 4T(n/4)+2cn \\
~ & \leq 4[2T(n/8) + cn/4] + 2cn = 8T(n/8)+3cn \\
~ & \leq 8[2T(n/16) + cn/8] + 3cn = 16T(n/16)+4cn \\
~ & \vdots
\end{align*}

A pattern is emerging. The general term is $T(n) \leq 2^k T(n/2^k) + kcn$. Plugging in $k=\lg n$ (the height of the recursion tree), we get $T(n) \leq nT(1) + cn \lg n = O(n \lg n)$.\\
\newline
Now we verify $O(n \lg n)$ using the substitution method:\\
\newline
Prove that
\begin{align*}
T(n) \leq c(n \lg n)
\end{align*}
Assume that
\begin{align*}
T(\frac{n}{2}) \leq c(\frac{n}{2} \lg \frac{n}{2})
\end{align*}
By substitution
\begin{align*}
T(n) & \leq 2(c(\frac{n}{2} \lg \frac{n}{2})) +  n \\
~ & = cn \lg \frac{n}{2} + n \\
~ & = cn \lg n - cn \lg 2 + n \\
~ & \leq cn \lg n \quad \text{holds for} \, c \geq 1 \\
\end{align*}

\item $T(n) = 3T(n/2) + n$ 

Insert your solution here

\item $T(n) = T(n/2) + n^2$

Insert your solution here

\item $T(n) = 4T(n/2 + 2) + n$

Insert your solution here

\item $T(n) = 2T(n-1) + 1$

Insert your solution here

\item $T(n) = T(n-1) + T(n/2) + n$

Insert your solution here

\end{itemize}

\subsection{Master Method}

Use the master method to give tight asymptotic $\Theta$ bounds:

\begin{itemize}

\item Example: $T(n) = 2T(n/2) + n$

Cost at leaves: $\Theta(n^{\log_b a}): n^{\log_2 2}=n=\Theta(n)$ \\
Cost per depth: $f(n)=n$ \\
Case 2: $f(n) = \Theta(n^{\log_b a})$: $f(n) = \Theta(n)$ \\
Solution: $\Theta(n^{\log_b a} \lg n) = \Theta(n \lg n)$

\item $T(n) = 2T(n/4) + 1$

Insert your solution here

\item $T(n) = 2T(n/4) + n$

Insert your solution here

\item $T(n) = 2T(n/2 + 17) + n$

Insert your solution here

\item $T(n) = 4T(n/2) + 2^n$

Insert your solution here

\item $T(n) = T(3n/4) + \sqrt{n}$

Insert your solution here

\end{itemize}

\section{Sorting}

\subsection{Plots}

Insert your plots here

\subsection{Discussion}

Insertion sort is faster than merge sort on small arrays up to size 8.
Merge sort involves copying the array which takes linear time, this might be why. 2*n > n^2 iff etc kind of thing. (need to be more precise).

One would think that Hybridsort would therefore be fastest by applying insertionsort when array small enough and merge sort otherwise, and results are above. 

But we can do better: since merge sort breaks down array into smaller arrays, and for sufficiently small arrays, insertion sort is better, maybe insertion sort should always be used 'towards the end' of hybrid sort to achieve better results. see plot above.

We can still do better: it's not because insertion is better up to size 8 that it should be used as soon as size is 8. Maybe merge should break array down to bits of size 4, and insertion should act on that. there would be more breaking down, but maybe this is more than compensated by just how much faster insertion is PER AVERAGE ENTRY on arrays of size 4 vs arrays of size 8. results above.

\section{Integer Multiplication}

\subsection{Plots}

\noindent Insert your plots here

\subsection{Discussion}

\noindent Insert your discussion here

\subsection{Recurrences}

\subsubsection{Recursive Multiplication}

\noindent Insert your solution here

\subsubsection{Karatsuba Multiplication}

\noindent Insert your solution here

\end{document}
